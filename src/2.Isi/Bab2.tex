%-----------------------------------------------------------------------------%
\chapter{\babDua}
%-----------------------------------------------------------------------------%
\lipsum[3-3]

%-----------------------------------------------------------------------------%
\section{Sejarah Singkat Perusahaan}
%-----------------------------------------------------------------------------%

\lipsum[4-5]

Contoh sitasi lainnya menggunakan \verb|\cite| adalah saat kita mau mensitasi pekerjaan tentang \textit{Expert System} \cite{Djong2023AMethod} \textit{Blockchain} \cite{Christyono2021}, \textit{Fuzzy model} \cite{Widjaja2012}, \textit{machine learning} \cite{jain:2004} dan \textit{Dynamic Programming} \cite{bellman:1962}. Dokumen \latex~sangat mudah, seperti halnya membuat dokumen teks biasa. Ada 
beberapa perintah yang diawali dengan tanda '\bslash'. 
Seperti perintah \bslash\bslash~yang digunakan untuk memberi baris baru. 
Perintah tersebut juga sama dengan perintah \bslash newline. 
Pada bagian ini akan sedikit dijelaskan cara manipulasi teks dan 
perintah-perintah \latex~yang mungkin akan sering digunakan. 
Jika ingin belajar hal-hal dasar mengenai \latex, silahkan kunjungi: 

\begin{itemize}
	\item \url{http://frodo.elon.edu/tutorial/tutorial/}, atau
	\item \url{http://www.maths.tcd.ie/~dwilkins/LaTeXPrimer/}
\end{itemize}

\cite{Marszaek2014ModelingCandlesticks}
%-----------------------------------------------------------------------------%
\section{Visi dan Misi Perusahaan}
%-----------------------------------------------------------------------------%
Agar dapat menggunakan \latex~(pada konteks hanya sebagai pengguna), Anda 
tidak perlu banyak tahu mengenai hal-hal di dalamnya. 
Seperti halnya pembuatan dokumen secara visual (contohnya Open Office (OO) 
Writer), Anda dapat menggunakan \latex~dengan cara yang sama. 
Orang-orang yang menggunakan \latex~relatif lebih teliti dan terstruktur 
mengenai cara penulisan yang dia gunakan, \latex~memaksa Anda untuk seperti itu.  

Kembali pada bahasan utama, untuk mencoba \latex~Anda cukup men-\textit{download} 
kompiler dan IDE. Saya menyarankan menggunakan Texlive dan Texmaker. 
Texlive dapat di-\textit{download} dari \url{http://www.tug.org/texlive/}. 
Sedangkan Texmaker dapat di-\textit{download} dari 
\url{http://www.xm1math.net/texmaker/}. 
Untuk pertama kali, coba buka berkas thesis.tex dalam template yang Anda miliki 
pada Texmaker. 
Dokumen ini adalah dokumen utama. 
Tekan F6 (PDFLaTeX) dan Texmaker akan mengkompilasi berkas tersebut menjadi 
berkas PDF. 
Jika tidak bisa, pastikan Anda sudah meng-\textit{install} Texlive. 
Buka berkas tersebut dengan menekan F7. 
Hasilnya adalah sebuah dokumen yang sama seperti dokumen yang Anda baca saat ini. 


%-----------------------------------------------------------------------------%
\section{Struktur Organisasi Perusahaan}
%-----------------------------------------------------------------------------%
Hal pertama yang mungkin ditanyakan adalah bagaimana membuat huruf tercetak 
tebal, miring, atau memiliki garis bawah. 
Pada Texmaker, Anda bisa melakukan hal ini seperti halnya saat mengubah dokumen 
dengan OO Writer. 
Namun jika tetap masih tertarik dengan cara lain, ini dia: 

\begin{itemize}
	\item \bo{Bold} \\
		Gunakan perintah \bslash textbf$\lbrace\rbrace$ atau 
		\bslash bo$\lbrace\rbrace$. 
	\item \f{Italic} \\
		Gunakan perintah \bslash textit$\lbrace\rbrace$ atau 
		\bslash f$\lbrace\rbrace$. 
	\item \underline{Underline} \\
		Gunakan perintah \bslash underline$\lbrace\rbrace$.
	\item $\overline{Overline}$ \\
		Gunakan perintah \bslash overline. 
	\item $^{superscript}$ \\
		Gunakan perintah \bslash $\lbrace\rbrace$. 
	\item $_{subscript}$ \\
		Gunakan perintah \bslash \_$\lbrace\rbrace$. 
\end{itemize}

Perintah \bslash f dan \bslash bo hanya dapat digunakan jika package 
uithesis digunakan untuk panduan. 

Struktur organisasi perusahaan PT XYZ dapat dilihat pada Gambar \ref{fig_struktur_organisasi}.

% %-----------------------------------------------------------------------------%
% \section{Memasukan Gambar}
% %-----------------------------------------------------------------------------%
% Setiap gambar dapat diberikan caption dan diberikan label. Label dapat 
% digunakan untuk menunjuk gambar tertentu. 
% Jika posisi gambar berubah, maka nomor gambar juga akan diubah secara 
% otomatis. 
% Begitu juga dengan seluruh referensi yang menunjuk pada gambar tersebut. 
% Contoh sederhana adalah \pic~\ref{fig:testGambar}. 
% Silahkan lihat code \latex~dengan nama bab2.tex untuk melihat kode lengkapnya. 
% Harap diingat bahwa caption untuk gambar selalu terletak dibawah gambar. 

\begin{figure}
	\centering
	\fbox{\includegraphics[width=0.90\textwidth]{assets/pics/fig_struktur-organisasi.jpg}}
	\caption{Struktur organisasi perusahaan PT XYZ judul gambar yang panjang lebih dari satu baris}
	\vspace{-1em}
	{\small Sumber: \cite{Widjaja2002a}}
	\label{fig_struktur_organisasi}
\end{figure}


