%-----------------------------------------------------------------------------%
\chapter{\babSatu}
%-----------------------------------------------------------------------------%

%-----------------------------------------------------------------------------%
\section{Latar Belakang Masalah}
%-----------------------------------------------------------------------------%
PT Ganda Visi Jayatama merupakan perusahaan yang bergerak di bidang penyediaan 
jasa pengembangan dan pembuatan perangkat lunak. Seiring bertambahnya jumlah 
karyawan dan divisi pada tahun 2023, kompleksitas organisasi meningkat dan 
muncul kebutuhan akan sistem pengelolaan sumber daya manusia yang efisien, 
terintegrasi, dan transparan.

Sistem manual yang sebelumnya digunakan terbukti kurang efisien dalam 
mengelola data absensi, cuti, struktur organisasi, dan penggajian. Oleh karena itu, 
perusahaan mulai mengembangkan CHRIS (Concise Human Resource Internal System), 
sebuah sistem berbasis web yang awalnya berfokus pada pencatatan absensi dan 
pengajuan cuti. Namun, seiring pertumbuhan organisasi, kebutuhan sistem berkembang 
lebih jauh mencakup hierarki supervisi dan sistem penggajian yang terotomatisasi.

Penelitian sebelumnya menekankan pentingnya pengembangan sistem informasi 
kepegawaian berbasis digital untuk mengatasi kompleksitas data karyawan, 
meningkatkan akurasi data, dan mendukung pengambilan keputusan strategis 
dalam manajemen HRD \cite{zykov2006hris}. Dalam hal ini, CHRIS dirancang untuk 
mengimplementasikan modul-modul penting seperti struktur hierarki antarpegawai, 
pengelolaan pengajuan cuti berdasarkan hubungan atasan-bawahan, serta modul 
payroll untuk slip gaji yang terintegrasi dengan data kehadiran dan status kerja 
pegawai.

Permasalahan dalam penggajian yang dilakukan secara terpisah dari data absensi 
juga ditemukan dalam banyak perusahaan berkembang. Sistem penggajian yang tidak 
terotomatisasi rawan kesalahan dan keterlambatan \cite{aina2025payroll}. Oleh karena itu, salah satu 
pengembangan penting yang dilakukan selama magang adalah penambahan modul 
payroll yang mampu menghitung gaji berdasarkan data karyawan secara otomatis 
dan terintegrasi dengan slip gaji dan tunjangan yang dimiliki.

Selain itu, struktur organisasi yang tidak didefinisikan secara digital 
menyulitkan proses pelaporan dan pengajuan cuti. Implementasi sistem hierarki 
digital dengan konsep \textit{hierarchy tree} terbukti memperbaiki alur komunikasi 
organisasi dan memperjelas relasi kerja antarpegawai \cite{kumar2024tree}.

Sistem ini dikembangkan menggunakan arsitektur PERN Stack (PostgreSQL, Express.js, 
React.js, Node.js). Pemilihan PostgreSQL didasarkan pada kemampuannya menangani 
relasi kompleks dan transaksi skala enterprise, sesuai dengan standar yang digunakan 
dalam pengembangan sistem informasi perusahaan modern \cite{arnold2019hrdbms}.

Sebagai Backend Engineer yang melanjutkan pengembangan sistem CHRIS, saya telah 
berkontribusi pada beberapa pengembangan krusial:

\begin{enumerate}
    \item \textbf{Pembuatan struktur \textit{Hierarchy Tree}}, yang berfungsi 
    untuk menetapkan struktur supervisi antarpegawai dan mempermudah logika 
    bisnis seperti persetujuan cuti.
    
    \item \textbf{Pengembangan modul Payroll dan Slip Gaji}, yang secara otomatis 
    menghitung gaji berdasarkan data status kepegawaian dan kehadiran, 
    meningkatkan efisiensi dan mengurangi kesalahan.
    
    \item \textbf{Refactor dan validasi ulang sistem User Management}, termasuk 
    pengelolaan data pegawai dan status pekerjaan untuk mendukung integrasi data 
    secara menyeluruh.
\end{enumerate}

Ketiga kontribusi tersebut merupakan bagian penting dalam mewujudkan sistem 
pengelolaan sumber daya manusia yang adaptif terhadap pertumbuhan perusahaan 
dan selaras dengan praktik terbaik pengembangan sistem informasi kepegawaian.

% %-----------------------------------------------------------------------------%
% \section{Permasalahan}
% %-----------------------------------------------------------------------------%
% Pada bagian ini akan dijelaskan mengenai definisi permasalahan 
% yang \saya~hadapi dan ingin diselesaikan serta asumsi dan batasan 
% yang digunakan dalam menyelesaikannya.


%-----------------------------------------------------------------------------%
\section{Maksud dan Tujuan Kerja Magang}
%-----------------------------------------------------------------------------%
Program kerja magang ini dilaksanakan dengan maksud untuk menerapkan berbagai 
\textit{hardskill} dan \textit{softskill} yang telah diperoleh selama masa 
perkuliahan ke dalam lingkungan kerja profesional. Adapun tujuan dari program 
kerja magang ini adalah untuk memperluas dan memperdalam \textit{hardskill} 
melalui berbagai tugas yang diemban, serta mengembangkan \textit{softskill} 
dalam berkoordinasi dan bekerja sama sebagai bagian dari tim. Secara spesifik, 
tujuan pelaksanaan kerja magang ini adalah untuk berkontribusi dalam 
pengembangan \textit{Human Resource Information System} (HRIS) pada 
PT Ganda Visi Jayatama.



%-----------------------------------------------------------------------------%
\section{Waktu dan Prosedur Pelaksanaan Kerja Magang}
%-----------------------------------------------------------------------------%

Pelaksanaan kerja magang berlangsung dari tanggal 13 Januari 2025 sampai dengan 
13 Juli 2025 berdasarkan kontrak kerja yang telah disepakati antara pihak 
perusahaan dan penulis. Selama periode magang ini, penulis dibimbing oleh 
seorang pembimbing lapangan yaitu Bapak Edo Setiawan yang 
menjabat sebagai Head Of Development di PT Ganda Visi Jayatama. Jadwal 
kerja magang di PT Ganda Visi Jayatama diatur sebagai berikut:

\begin{enumerate}
    \item Aktivitas kerja magang dilaksanakan setiap hari Senin hingga Jumat, 
    dengan jam kerja mulai pukul 09.00 WIB sampai dengan 18.00 WIB.
    \item Pelaksanaan kerja magang dilakukan secara Work From Office (WFO).
\end{enumerate}

Selama menjalani program kerja magang, penulis mengikuti sejumlah prosedur 
yang telah ditetapkan, antara lain:

\begin{enumerate}
    \item Mengikuti sesi orientasi (onboarding) pada minggu pertama kerja magang.
    \item Melakukan presensi harian dengan mencatat tugas yang telah diselesaikan 
    pada hari sebelumnya (yesterday tasks), rencana aktivitas untuk hari ini 
    (today tasks), serta kendala yang dihadapi dalam pengerjaan sebelumnya (blocking).
    \item Berpartisipasi dalam rapat mingguan yang diadakan setiap hari Jumat 
    untuk membahas perkembangan proyek HRIS yang sedang dikerjakan.
    \item Menghadiri pertemuan bulanan untuk mendiskusikan pengembangan 
    boilerplate perusahaan.
    \item Berkomunikasi dengan sesama karyawan melalui platform Discord.
\end{enumerate}
