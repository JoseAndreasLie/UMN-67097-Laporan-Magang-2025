%-----------------------------------------------------------------------------%
\chapter{\babSatu}
%-----------------------------------------------------------------------------%

%-----------------------------------------------------------------------------%
\section{Latar Belakang Masalah}
%-----------------------------------------------------------------------------%
\lipsum[1-2]

% %-----------------------------------------------------------------------------%
% \section{Permasalahan}
% %-----------------------------------------------------------------------------%
% Pada bagian ini akan dijelaskan mengenai definisi permasalahan 
% yang \saya~hadapi dan ingin diselesaikan serta asumsi dan batasan 
% yang digunakan dalam menyelesaikannya.


%-----------------------------------------------------------------------------%
\section{Maksud dan Tujuan Kerja Magang}
%-----------------------------------------------------------------------------%
Program kerja magang ini dilaksanakan dengan maksud untuk menerapkan berbagai 
\textit{hardskill} dan \textit{softskill} yang telah diperoleh selama masa 
perkuliahan ke dalam lingkungan kerja profesional. Adapun tujuan dari program 
kerja magang ini adalah untuk memperluas dan memperdalam \textit{hardskill} 
melalui berbagai tugas yang diemban, serta mengembangkan \textit{softskill} 
dalam berkoordinasi dan bekerja sama sebagai bagian dari tim. Secara spesifik, 
tujuan pelaksanaan kerja magang ini adalah untuk berkontribusi dalam 
pengembangan \textit{Human Resource Information System} (HRIS) pada 
PT Ganda Visi Jayatama.



%-----------------------------------------------------------------------------%
\section{Waktu dan Prosedur Pelaksanaan Kerja Magang}
%-----------------------------------------------------------------------------%

Pelaksanaan kerja magang berlangsung dari tanggal 13 Januari 2025 sampai dengan 
13 Juli 2025 berdasarkan kontrak kerja yang telah disepakati antara pihak 
perusahaan dan penulis. Selama periode magang ini, penulis dibimbing oleh 
seorang pembimbing lapangan yaitu Bapak Edo Setiawan yang 
menjabat sebagai Head Of Development di PT Ganda Visi Jayatama. Jadwal 
kerja magang di PT Ganda Visi Jayatama diatur sebagai berikut:

\begin{enumerate}
    \item Aktivitas kerja magang dilaksanakan setiap hari Senin hingga Jumat, 
    dengan jam kerja mulai pukul 09.00 WIB sampai dengan 18.00 WIB.
    \item Pelaksanaan kerja magang dilakukan secara Work From Office (WFO).
\end{enumerate}

Selama menjalani program kerja magang, penulis mengikuti sejumlah prosedur 
yang telah ditetapkan, antara lain:

\begin{enumerate}
    \item Mengikuti sesi orientasi (onboarding) pada minggu pertama kerja magang.
    \item Melakukan presensi harian dengan mencatat tugas yang telah diselesaikan 
    pada hari sebelumnya (yesterday tasks), rencana aktivitas untuk hari ini 
    (today tasks), serta kendala yang dihadapi dalam pengerjaan sebelumnya (blocking).
    \item Berpartisipasi dalam rapat mingguan yang diadakan setiap hari Jumat 
    untuk membahas perkembangan proyek HRIS yang sedang dikerjakan.
    \item Menghadiri pertemuan bulanan untuk mendiskusikan pengembangan 
    boilerplate perusahaan.
    \item Berkomunikasi dengan sesama karyawan melalui platform Discord.
\end{enumerate}
