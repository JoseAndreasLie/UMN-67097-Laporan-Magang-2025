%-----------------------------------------------------------------------------%
\chapter{\babSatu}
%-----------------------------------------------------------------------------%

%-----------------------------------------------------------------------------%
\section{Latar Belakang Masalah}
%-----------------------------------------------------------------------------%
PT Ganda Visi Jayatama merupakan perusahaan yang bergerak di bidang penyediaan 
jasa pengembangan dan pembuatan perangkat lunak. Sejak tahun 2023, perusahaan 
mulai memperluas jumlah karyawan dan divisi pekerjaan yang ada. Seiring dengan 
pertumbuhan ini, struktur organisasi menjadi lebih kompleks dan sistem pengelolaan 
sumber daya manusia yang ada semakin tidak memadai, khususnya dalam hal 
efisiensi, transparansi, dan manajemen hirarki kepegawaian. Oleh karena itu, 
perusahaan mengembangkan CHRIS (Concise Human Resource Internal System), sebuah 
sistem digital yang awalnya dirancang untuk mempermudah pencatatan absensi dan 
cuti. Digitalisasi sistem ini menjadi krusial karena metode manual terbukti tidak 
efektif dalam pengelolaan data kepegawaian yang semakin kompleks [1].

Pada versi pertamanya, CHRIS tidak hanya menyediakan sistem absensi dan perizinan 
cuti, tetapi juga mencakup fitur laporan Daily Stand-Up Meeting (DSM). DSM 
merupakan praktik penting dalam metodologi agile seperti Scrum dan Extreme 
Programming (XP), yang membantu meningkatkan komunikasi tim dalam proyek 
perangkat lunak. Pertemuan ini berlangsung singkat setiap pagi dan bertujuan 
untuk menyampaikan perkembangan pekerjaan secara efisien [2]. Namun, seiring 
berkembangnya perusahaan, muncul kebutuhan untuk memperjelas struktur supervisi 
antar pegawai yang belum terakomodasi dalam sistem yang ada.

Selain itu, pertumbuhan jumlah karyawan juga menghasilkan tantangan baru dalam 
pengelolaan kompensasi dan tunjangan. Sistem yang ada belum terintegrasi dengan 
proses penggajian, sehingga perhitungan gaji masih dilakukan secara terpisah 
dengan data kehadiran dan performa karyawan. Hal ini sering menyebabkan 
inefisiensi dan potensi kesalahan dalam penggajian [3]. Pengelolaan berbagai 
jenis tunjangan karyawan juga menjadi semakin kompleks dan membutuhkan sistem 
yang lebih terstruktur.

Pengembangan CHRIS direncanakan akan terus berlanjut dengan menambahkan 
fitur manajemen proyek dan manajemen tugas pada versi-versi selanjutnya. 
Fitur-fitur ini diadaptasi dari aplikasi ClickUp, yang populer dalam pengelolaan 
proyek dan tugas. Namun, versi gratis ClickUp memiliki keterbatasan dalam jumlah 
proyek yang dapat dibuat [4][5]. Oleh karena itu, CHRIS versi kedua difokuskan 
untuk menghadirkan solusi yang lebih komprehensif dan terintegrasi melalui 
pengembangan aplikasi web internal perusahaan.

CHRIS dibangun menggunakan teknologi berbasis web untuk meningkatkan usability 
dan efisiensi [6]. Proyek ini menggunakan PERN Stack yang terdiri dari PostgreSQL, 
Express, React, dan Node.js untuk membangun sistem full-stack yang andal dan 
mendukung operasi CRUD. PostgreSQL dipilih sebagai basis data karena kemampuannya 
dalam menangani fitur NoSQL, transaksi, serta kesesuaiannya dengan standar 
industri [7]. Arsitektur ini memberikan fleksibilitas yang dibutuhkan untuk 
mengembangkan fitur-fitur baru yang saling terintegrasi dalam sistem.

Sebagai Backend Engineer yang melanjutkan pengembangan sistem CHRIS, saya telah 
menambahkan dan mengembangkan tiga fitur penting untuk mengatasi tantangan 
yang dihadapi perusahaan dalam pengelolaan sumber daya manusia:

\begin{enumerate}
    \item \textbf{Pembuatan sistem Hierarchy Tree untuk atasan langsung}, 
    memungkinkan struktur supervisi dan hubungan antarpegawai lebih jelas, 
    sehingga alur persetujuan cuti dan pelaporan kinerja menjadi lebih efisien 
    dan terstruktur.
    
    \item \textbf{Integrasi Payroll}, untuk menghubungkan sistem gaji dengan 
    data kehadiran, performa, dan informasi kepegawaian lainnya secara otomatis, 
    mengurangi kesalahan perhitungan dan meningkatkan transparansi dalam proses 
    penggajian.
    
    \item \textbf{Penambahan sistem User Allowance yang terintegrasi dengan 
    Slip Gaji}, untuk mendukung pencatatan dan perhitungan berbagai jenis tunjangan 
    karyawan secara rinci, memudahkan departemen HR dalam mengelola kompensasi 
    karyawan sesuai dengan kebijakan perusahaan.
\end{enumerate}

Ketiga fitur tersebut saling terintegrasi untuk menciptakan sistem pengelolaan 
sumber daya manusia yang lebih komprehensif, transparan, dan efisien, sesuai 
dengan kebutuhan PT Ganda Visi Jayatama yang terus berkembang.

% %-----------------------------------------------------------------------------%
% \section{Permasalahan}
% %-----------------------------------------------------------------------------%
% Pada bagian ini akan dijelaskan mengenai definisi permasalahan 
% yang \saya~hadapi dan ingin diselesaikan serta asumsi dan batasan 
% yang digunakan dalam menyelesaikannya.


%-----------------------------------------------------------------------------%
\section{Maksud dan Tujuan Kerja Magang}
%-----------------------------------------------------------------------------%
Program kerja magang ini dilaksanakan dengan maksud untuk menerapkan berbagai 
\textit{hardskill} dan \textit{softskill} yang telah diperoleh selama masa 
perkuliahan ke dalam lingkungan kerja profesional. Adapun tujuan dari program 
kerja magang ini adalah untuk memperluas dan memperdalam \textit{hardskill} 
melalui berbagai tugas yang diemban, serta mengembangkan \textit{softskill} 
dalam berkoordinasi dan bekerja sama sebagai bagian dari tim. Secara spesifik, 
tujuan pelaksanaan kerja magang ini adalah untuk berkontribusi dalam 
pengembangan \textit{Human Resource Information System} (HRIS) pada 
PT Ganda Visi Jayatama.



%-----------------------------------------------------------------------------%
\section{Waktu dan Prosedur Pelaksanaan Kerja Magang}
%-----------------------------------------------------------------------------%

Pelaksanaan kerja magang berlangsung dari tanggal 13 Januari 2025 sampai dengan 
13 Juli 2025 berdasarkan kontrak kerja yang telah disepakati antara pihak 
perusahaan dan penulis. Selama periode magang ini, penulis dibimbing oleh 
seorang pembimbing lapangan yaitu Bapak Edo Setiawan yang 
menjabat sebagai Head Of Development di PT Ganda Visi Jayatama. Jadwal 
kerja magang di PT Ganda Visi Jayatama diatur sebagai berikut:

\begin{enumerate}
    \item Aktivitas kerja magang dilaksanakan setiap hari Senin hingga Jumat, 
    dengan jam kerja mulai pukul 09.00 WIB sampai dengan 18.00 WIB.
    \item Pelaksanaan kerja magang dilakukan secara Work From Office (WFO).
\end{enumerate}

Selama menjalani program kerja magang, penulis mengikuti sejumlah prosedur 
yang telah ditetapkan, antara lain:

\begin{enumerate}
    \item Mengikuti sesi orientasi (onboarding) pada minggu pertama kerja magang.
    \item Melakukan presensi harian dengan mencatat tugas yang telah diselesaikan 
    pada hari sebelumnya (yesterday tasks), rencana aktivitas untuk hari ini 
    (today tasks), serta kendala yang dihadapi dalam pengerjaan sebelumnya (blocking).
    \item Berpartisipasi dalam rapat mingguan yang diadakan setiap hari Jumat 
    untuk membahas perkembangan proyek HRIS yang sedang dikerjakan.
    \item Menghadiri pertemuan bulanan untuk mendiskusikan pengembangan 
    boilerplate perusahaan.
    \item Berkomunikasi dengan sesama karyawan melalui platform Discord.
\end{enumerate}
