%-----------------------------------------------------------------------------%
\chapter{\babSatu}
%-----------------------------------------------------------------------------%

%-----------------------------------------------------------------------------%
\section{Latar Belakang Masalah}
\todo{KERJAKAN, MINIMAL 10 SITASI.}
%-----------------------------------------------------------------------------%
PT Ganda Visi Jayatama merupakan perusahaan yang bergerak di bidang penyediaan jasa pengembangan dan pembuatan perangkat lunak. Seiring bertambahnya jumlah karyawan dan divisi pada tahun 2023, kompleksitas organisasi meningkat dan muncul kebutuhan akan sistem pengelolaan sumber daya manusia yang efisien, terintegrasi, dan transparan.

Sistem manual yang sebelumnya digunakan terbukti kurang efisien dalam mengelola data absensi, cuti, struktur organisasi, dan penggajian. Oleh karena itu, perusahaan mulai mengembangkan CHRIS (Concise Human Resource Internal System), sebuah sistem berbasis \textit{web} yang awalnya berfokus pada pencatatan absensi dan pengajuan cuti. Namun, seiring pertumbuhan organisasi, kebutuhan sistem berkembang lebih jauh mencakup hierarki supervisi dan sistem penggajian yang terotomatisasi.

Era digitalisasi telah mengubah paradigma pengelolaan sumber daya manusia dalam organisasi modern. Sistem Informasi Sumber Daya Manusia (HRIS) menjadi komponen vital dalam mendukung efektivitas dan efisiensi operasional perusahaan. Penelitian menunjukkan bahwa implementasi HRIS memberikan dampak positif signifikan terhadap kinerja organisasi melalui peningkatan efisiensi dan efektivitas pengelolaan kepegawaian \cite{khashman2016impact}.

Kompleksitas pengelolaan data karyawan dalam skala besar memerlukan sistem yang terintegrasi dan terautomatisasi. HRIS berfungsi sebagai platform digital yang membantu organisasi dalam mengelola sumber daya manusia dengan memusatkan, mengorganisir, dan mengotomatisasi proses-proses yang terkait dengan fungsi HR \cite{abdullah2023impact}. Sistem ini tidak hanya mengatasi keterbatasan sistem manual yang rentan error, tetapi juga memberikan dukungan strategis dalam pengambilan keputusan manajemen.

Pentingnya pengembangan HRIS semakin diperkuat oleh dampak globalisasi dan teknologi yang mendorong organisasi untuk mengadopsi sistem informasi dalam berbagai fungsi departemen. Penelitian menunjukkan bahwa HRIS membantu manajemen senior dalam mengidentifikasi kebutuhan tenaga kerja untuk memenuhi rencana bisnis jangka panjang organisasi, sekaligus mendukung aktivitas identifikasi calon karyawan potensial dan penciptaan program pengembangan talenta \cite{ozel2012importance}.

Efektivitas implementasi HRIS telah terbukti dalam berbagai konteks industri, termasuk sektor rumah sakit swasta, di mana sistem ini berkontribusi signifikan terhadap peningkatan kinerja organisasi. Studi empiris menunjukkan korelasi positif yang kuat antara pentingnya HRIS dengan kinerja organisasi, dengan nilai korelasi mencapai R = 0.889 \cite{khashman2016impact}.

% Sistem ini dikembangkan menggunakan arsitektur PERN Stack (PostgreSQL, Express.js, React.js, Node.js). Pemilihan PostgreSQL didasarkan pada kemampuannya menangani relasi kompleks dan transaksi skala enterprise, sesuai dengan standar yang digunakan dalam pengembangan sistem informasi perusahaan modern \cite{arnold2019hrdbms}.


    % KONTEN BAB 3
    % Sebagai Backend Engineer yang melanjutkan pengembangan sistem CHRIS, penulis telah 
    % berkontribusi pada beberapa pengembangan krusial:

    % \begin{enumerate}
    %     \item \textbf{Pembuatan struktur \textit{Hierarchy Tree}}, yang berfungsi 
    %     untuk menetapkan struktur supervisi antarpegawai dan mempermudah logika 
    %     bisnis seperti persetujuan cuti.
        
    %     \item \textbf{Pengembangan modul Payroll dan Slip Gaji}, yang secara otomatis 
    %     menghitung gaji berdasarkan data status kepegawaian dan kehadiran, 
    %     meningkatkan efisiensi dan mengurangi kesalahan.
        
    %     \item \textbf{Refactor dan validasi ulang sistem User Management}, termasuk 
    %     pengelolaan data pegawai dan status pekerjaan untuk mendukung integrasi data 
    %     secara menyeluruh.
    % \end{enumerate}

    % Ketiga kontribusi tersebut merupakan bagian penting dalam mewujudkan sistem 
    % pengelolaan sumber daya manusia yang adaptif terhadap pertumbuhan perusahaan 
    % dan selaras dengan praktik terbaik pengembangan sistem informasi kepegawaian.

% %-----------------------------------------------------------------------------%
% \section{Permasalahan}
% %-----------------------------------------------------------------------------%
% Pada bagian ini akan dijelaskan mengenai definisi permasalahan 
% yang \saya~hadapi dan ingin diselesaikan serta asumsi dan batasan 
% yang digunakan dalam menyelesaikannya.


%-----------------------------------------------------------------------------%
\section{Maksud dan Tujuan Kerja Magang}

\todo{KERJAKAN, HARUS MENJAWAB JUDUL.}
%-----------------------------------------------------------------------------%


% MAKSUD DAN TUJUAN HARUS MENJAWAB JUDUL
% Program kerja magang ini dilaksanakan dengan maksud untuk menerapkan berbagai 
% \textit{hardskill} dan \textit{softskill} yang telah diperoleh selama masa 
% perkuliahan ke dalam lingkungan kerja profesional. Adapun tujuan dari program 
% kerja magang ini adalah untuk memperluas dan memperdalam \textit{hardskill} 
% melalui berbagai tugas yang diemban, serta mengembangkan \textit{softskill} 
% dalam berkoordinasi dan bekerja sama sebagai bagian dari tim. Secara spesifik, 
% tujuan pelaksanaan kerja magang ini adalah untuk berkontribusi dalam 
% pengembangan \textit{Human Resource Information System} (HRIS) pada 
% PT Ganda Visi Jayatama.





%-----------------------------------------------------------------------------%
\section{Waktu dan Prosedur Pelaksanaan Kerja Magang}
%-----------------------------------------------------------------------------%

Pelaksanaan kerja magang berlangsung dari tanggal 13 Januari 2025 sampai dengan 
13 Juli 2025 berdasarkan kontrak kerja yang telah disepakati dengan perusahaan. Selama periode magang ini dibimbing oleh seorang pembimbing lapangan yaitu Bapak Edo Setiawan yang menjabat sebagai Head Of Development di PT Ganda Visi Jayatama. Jadwal kerja magang di PT Ganda Visi Jayatama diatur sebagai berikut:

\begin{enumerate}
    \item Aktivitas kerja magang dilaksanakan setiap hari Senin hingga Jumat, 
    dengan jam kerja mulai pukul 09.00 WIB sampai dengan 18.00 WIB.
    \item Pelaksanaan kerja magang dilakukan secara \textit{Work From Office} (WFO).
\end{enumerate}

Selama menjalani program kerja magang, penulis mengikuti sejumlah prosedur yang telah ditetapkan, antara lain:

\begin{enumerate}
    \item Mengikuti sesi orientasi (\textit{onboarding}) pada minggu pertama kerja magang.
    \item Melakukan presensi harian dengan mencatat tugas yang telah diselesaikan 
    pada hari sebelumnya (\textit{yesterday tasks}), rencana aktivitas untuk hari ini 
    (\textit{today tasks}), serta kendala yang dihadapi dalam pengerjaan sebelumnya (\textit{blocking}).
    \item Berpartisipasi dalam rapat mingguan yang diadakan setiap hari Jumat 
    untuk membahas perkembangan proyek HRIS yang sedang dikerjakan.
    \item Menghadiri pertemuan bulanan untuk mendiskusikan pengembangan 
    \textit{boilerplate} perusahaan.
    \item Berkomunikasi dengan sesama karyawan melalui \textit{platform Discord}.
\end{enumerate}
