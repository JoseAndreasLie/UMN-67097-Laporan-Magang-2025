%-----------------------------------------------------------------------------%
\chapter*{\Judul}
%-----------------------------------------------------------------------------%
\singlespacing
\begin{center}
    
    \vspace{-4em}
    
    \penulis
    
	\bigskip
    
    \textbf{ABSTRAK}
    
\end{center}

% \chapter*{Abstrak}

\vspace*{0.2cm}
{
	\setlength{\parindent}{0pt}

	\bigskip
	\bigskip

	Laporan ini menjelaskan proses pengembangan dan penyempurnaan \textit{backend} dari sistem informasi kepegawaian CHRIS (Concise \textit{Human Resource Information System}) di PT Ganda Visi Jayatama. Fokus utama proyek ini adalah penambahan fitur baru seperti otentikasi biometrik pada CHRISM (CHRIS \textit{Mobile}), sistem hierarki persetujuan cuti berbasis pohon, serta pengembangan penuh modul \textit{Payroll}. Modul \textit{User Management} juga diperbaiki melalui peningkatan validasi input pada formulir. Sistem dikembangkan menggunakan teknologi \textit{PERN Stack} (\textit{PostgreSQL}, \textit{Express.js}, \textit{React.js}, dan \textit{Node.js}) dengan metodologi \textit{Agile}. Selama proyek, 17 \textit{endpoint API} baru dikembangkan dan 7 \textit{endpoint} yang sudah ada ditingkatkan, menghasilkan total 24 \textit{endpoint} yang aktif digunakan. Pengujian dilakukan secara manual. Hasil dari pengembangan ini mencakup peningkatan efisiensi dan akurasi penggajian, keamanan akses sistem melalui biometrik, serta alur persetujuan cuti yang lebih fleksibel. Tantangan teknis seperti kode yang tidak sesuai \textit{best practice}, penggunaan nilai \textit{hardcoded}, dan pengolahan data historis yang kompleks diatasi melalui refactoring, penambahan struktur data, dan optimasi proses migrasi.

	\bigskip
 
% Kata kunci urut abjad
% 3 – 5 kata kunci
	\textbf{Kata Kunci}: \textit{Backend development, Biometric, Payroll, Human resource information system, Hierarchical structure.}	
}

\onehalfspacing