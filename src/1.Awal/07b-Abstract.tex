%-----------------------------------------------------------------------------%
\chapter*{\MakeUppercase{\textit{\judulInggris}}}
%-----------------------------------------------------------------------------%
\singlespacing
\begin{center}
    
    \vspace{-4em}
    
    \penulis
    
	\bigskip
    
    \textit{\textbf{ABSTRACT}}
    
\end{center}

% \chapter*{Abstrak}

\vspace*{0.2cm}
{
	\setlength{\parindent}{0pt}

	\bigskip
	\bigskip


	This report presents the backend development and enhancement of the CHRIS (Concise Human Resource Information System) used by PT Ganda Visi Jayatama. The development focused on extending existing modules and integrating new features, such as biometric authentication for CHRISM (CHRIS Mobile), a tree-based hierarchy system for the leave permit module, and the implementation of a fully new Payroll module. Additionally, improvements were made to the User Management module by strengthening input validation.

	The system was developed using the PERN Stack (PostgreSQL, Express.js, React.js, and Node.js) under an Agile development methodology. In total, 17 new API endpoints were developed, along with the enhancement of 7 existing ones, resulting in 24 functional endpoints across the system. All testing was conducted manually.

	The improvements provided significant impact in operational efficiency—automated salary calculation through the Payroll module, increased security via biometric authentication, and a more dynamic leave approval process through hierarchical supervision. Challenges encountered during the project included codebase that did not follow best practices, hardcoded values, and complex historical data handling. These were resolved through refactoring, additional transaction tables, and optimization of data migration processes.
	
	\bigskip
 
% keywords in alphabetical order
% 3 – 5 keywords
	\textit{\textbf{Keywords}: Backend development, Biometric, Payroll, Human resource information system, Hierarchical structure.}	
}

\onehalfspacing